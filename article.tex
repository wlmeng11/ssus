\documentclass[conference]{IEEEtran}
\IEEEoverridecommandlockouts
% The preceding line is only needed to identify funding in the first footnote. If that is unneeded, please comment it out.
\usepackage{cite}
\usepackage{amsmath,amssymb,amsfonts}
\usepackage{algorithmic}
\usepackage{graphicx}
\usepackage{textcomp}
\usepackage{xcolor}
\def\BibTeX{{\rm B\kern-.05em{\sc i\kern-.025em b}\kern-.08em
    T\kern-.1667em\lower.7ex\hbox{E}\kern-.125emX}}
\begin{document}

\title{The interest of single-sensor ultrasound imaging}

\author{\IEEEauthorblockN{Luc Jonveaux}
\IEEEauthorblockA{\textit{Open source tinkerer} \\
\textit{Independant}\\
Milly, France\\
contact@un0rick.cc}

}

\maketitle

\begin{abstract}
This document is a model and instructions for \LaTeX.
This and the IEEEtran.cls file define the components of your paper [title, text, heads, etc.]. *CRITICAL: Do Not Use Symbols, Special Characters, Footnotes, 
or Math in Paper Title or Abstract.
\end{abstract}

\begin{IEEEkeywords}
ultrasound imaging, open source, hardware, ultrasound
\end{IEEEkeywords}

\section{Introduction}


 
 
Previous work was done on an arduino-like approach \cite{jonveaux_arduino-like_2017}.

\subsection{Ultrasound as part of medical imaging }

Ultrasound is:

* cost-efficiency. Signal generation uses a piezoelectric material, requiring much less than setting up for example a large magnetic field for an MRI scan. 
* relatively safe: as there are no ionizing radiation, ultrasound devices do not need special safety equipment like shields, or expensive supplies.
* light: on the sense that compared to other imaging modalities, devices can be made relatively portable
* dynamic


See Kurjak



\subsection{Open source medical hardware}

Open hardware lowers some barriers: having a full design under an open-source license provides the key to all to contribute and improve a design. This in turns allows a possible rapid spread of the design, customisation for specific uses, and ad-hoc modification. A design being open means that a higher number of contributors can help inspect and improve it.

Repairs are also easier with the code, with several types of impact (social, environmental and economics) that enhance the lifecycle of a product.

The open source movement has been facilitated by the numerous platforms and communication means, that allow these designs to be copied, but also for robust storage and distribution means. 


\cite{niezen_open-source_2016}


Open source hardware can be a disruptive tool on the medical device market. Of course, a caveat has to be made on the non-certification of most of these devices, which in turn would hint at open-source being a major tool in innovation, leaving the industrialisation of such devices to professionals. Shorter development cycles, even for hardware, with open source permits quick iterations over a product.


A study by Niezen \cite{niezen_open-source_2016} lists a selection of open-source medical projects. Most are working with Arduino or Raspberry Pi, showing the importance of existing projects and the way open-source allows for reuse of previous projects. Dedicated hardware is developped for this project, leaving the specifics of the project to be developped by the contributor.

\section{Ultrasound imaging}

A

\section{A-Mode imaging}


This modality can be used for example in image free systems for evaluation of vascular stiffness. \cite{joseph_artsenstouch_2015}


Probes \cite{carotenuto_very_2004}
Systems

but these have not been open-sourced.

\section{Acquisition modalities}



\subsection{Enveloppe extraction}

For FPGA
chang_novel_2007

\subsection{Pulse echo}


\subsubsection{NDE}

\cite{fritsch_full_nodate} presents a very interesting design for single element FPGA-based NDE design. It migrates traditionally analog functions, like filtering and envelope extraction to   the   digital   domain, as well as includes an EMI filteR. An interesting data compression scheme is also presented.


\subsection{Compressed sensing}

Traditional 2-D and 3D ultrasound require the use of hundreds of delicates sensors, with matching hardware. 

Ultrasound 2D arrays, or matrices, have developped available, however this also require more hardware (AFE, ADC units, cabling, etc..) and hence become more
complex and expensive to produce, and require heavy data processing.

@todo add \cite{Fedjajevs_mscthesis_14_09_2016} thesis

It appears that classical sampling is challenged by the signal processing "compressive sensing" field. This shows that most signals have a sparse representation - a finite sparse signal can be reconstructed from a small set of linear, non-adaptive measurements.

The recent development of compressive sensing shows that one can encode individual voxels through a 'chaotic' intermediary (eg a lense), and allows to design simple ultrasound imaging equipment that can provide 3D imaging using a single-element ultrasound sensor.  

Imaging is then offloaded to the processing domain supported by constantly growing electronic devices' capabilities (memory, GPU's).


This open doors to simpler hardware - and new applications \cite{kruizinga_compressive_2017}.


\cite{hua_compressed_2011}

\subsection{Time reversal focusing}

@adapt N.Etaix et.al [19] presented an idea of a low-profile acoustic imaging device with only one transmit/receive element. According to the theory, if the medium is reciprocal and the channel’s impulse response is known, the latter can be re-emitted in time-reversed order. Mathematically, this results in the response auto-convolution [20]. Knowing that the auto-convolution has a peak in the origin, focusing is effectively achieved. The time reversal result is equivalent to matched filtering – energy maximization at the desired location in space and time.

\subsection{Basics for US machine}

\subsubsection{specs from who}


ref kurjak

In 1986, the specifications for a General Purpose Ultrasound Scanner were as follows:

- Have linear and sector types
- Frequency of interest is 3.5MHz, possibly extensible to 5MHz.
- Sector angle of 40 degrees or more
- Depth of 18cm
- rate of 5 to 10Hz for sector scanning
- image to be frozen, displayed on a 512x512 image on 4 bits.

Minimal specifications for hardware can be infered:

- An ADC at at least 10MHz, running on 8bits
- 18cm means 240us per acquisition, or 2400 points at 10Msps. Details can be around 0.5mm, that's then  480 points per line. Let's consider 512 for the image size.
- Let's consider 60 degree. To have a 1mm resolution at 18cm means 180 lines per image, which will be fed into the 512x512 image. However, given the size of the image and characterics of the piezo, it seems reasonable to have 60 lines over 40 degrees.
- Raw images are then 60x512x8bits, that's a memory of 30kB+ before scan conversion.
- 60 lines of 240us means 14ms ofr an image, capping the framerate to close to 70fps - seems fine. However, mechanical overheads reduce this figure.



\subsubsection{key figures}

\subsection{Use case}


Even if relatively simple, the A-mode enables not imaging, but rather measurements, enabling the examination of paranasal sinuses, transkull fluid detection, sinus pathology, skeletal muscle detection in the guided wrist extension, automatic measurement of lumen diameter of carotid artery \cite{li_new_2014}, or bone porosit \cite{wahab_design_2016}.



\subsection{Results}

\subsection{3D imaging }

see augmented herickhoff_low-cost_2018

or 

poulsen_optical_2005.pdf

or 

lorenzo_experimental_2009 pour reperer dans l'espace



\subsection{Tomorgraphy}

\cite{kuzmin_fast_2016}

\subsection{Use case}

\subsubsection{Sleep Apnea}

\cite{weng_fpga-based_2015}

\section{B-Mode imaging}

\subsection{Basics for US machine}

\subsubsection{specs from who}


% iris.wpro.who.int/bitstream/handle/10665.1/5467/9290611219_eng.pdf

\cite{kurjak_use_1986}


\subsection{Data reduction}

soto-cajiga_fpga-based_2012

\subsubsection{key figures}

\subsection{Architecture}



Configurable hardware makes the system resilient to future changes.
Beyond this hardware designs can be adjusted without reprinting the circuit board.

FPGAs improve the ability for ultrasound imaging systems to create small form factor and high-performance products with reduced power consumption \cite{dusa_low_2014}. In the latter case, it enables the control of a 8 channels platform.
  
Others use  an  additional microcontroler is set up between the FPGA and the USB and provide configuration on the fly, ,providing access to the platform through Matlab. It can interface to an array using several Octal Ultrasound ADC/LNA/VGA/AAF ( AD9272 ) \cite{raj_microcontroller_2017}, \cite{raj_8051_2016}.


\subsection{Why FPGAs ?}

zhang_fpga_2012


\subsection{2D imaging .. }

\subsubsection{Sweeping means}


mechanical sweeping example \cite{svilainis_electronics_2014}.


One would note however, that for power-uses, plane wave imaging, made possible because of GPUs \cite{hewener_mobile_2015}, would limit the formation of an image to a single pulse/receive event.


With high frequency transducers (c. 50 MHz), imaging transducers are relatively smaller, which makes mechanical scanning a great solution. This implies however strong positionning control and motors \cite{carotenuto_very_2004}. In this article, a piezoelectric	 motor is used in conjunction with an optical encoder. 


\subsubsection{Arrays}


However, the need heavier setup, for example to use up to 128 channels setups \cite{assef_flexible_2015}.




\subsection{MSAS/MFFS}

\cite{ylitalo_ultrasound_1994}

\cite{heuvel_development_2017}

nikolov_fast_2008


In 1974, \cite{burckhardt_experimental_1974} showed that using a synthetic aperture design as an hybrid  between holography and B scan offers a higher lateral resolution as well as better detection of tilted specular reflectors.

\subsection{Use case}

\subsubsection{Ultrasound capsule}

Swallowable capsule : capsule endoscopy using  high frequency microultrasound (µUS). This is a great example of miniaturisation fitting all in a  10 mm diameter ×30 mm long capsule , with all the constraints discussed above \cite{cox_ultrasound_2017}.

further develops mechanics used in such a capsule \cite{wang_development_2017}.

\subsubsection{Fingers movement}

\cite{sikdar_novel_2014}

\subsection{Results}


\subsection{To go further}

\cite{kim_smart-phone_2017} proposes a design usable on a smartphone, connected through USB. Plane-wave acquisition, mentioned earlier \cite{hewener_mobile_2015} also enables smarter captures.


 
Proof of concept for a portalbe 32-channel ultrasound unit \cite{kim_single_2012}

\section{Documentation}

\subsection{Why documenting}

Open source = documentation

\subsection{How to document}

On this project, a set of documenting tools was created. Documentation is based on the constitution of a project body of knowledge, which consists in code files, documentation files, images, and other files being tagged with structured information.

The concept used has been to store information only once, and closest to the source of the documentation. The main objective is to avoid redundancy and to allow for tracking outdated information, for elimination.

Tags cover for example date and time, type of file, experiment related to the file, probes being used, description of the file, author, ...  "Template files", that have "includes" of parts of documentation, also dynamically generate files.

This way, a script parses the whole, possibly unstructured repository and builds a body of knowledge, which in turn is used to generate static pages, that can cover:

- List of experiment and descriptions
- List of probes
- List of modules
- Jupyter notebooks
- Readme pages

The structured data and produced static pages are also pushed on a gitbook, so that documentation is easily accessible and searchable. Even Jupyter notebooks are converted to markdown for ease of browsing.

Moreover, a worklog is used to generate a lab-log of research, published on a github-page structure. Presentations are dynamically generated as well from markdown files, and a log of the documentation compilation is generated, tracking broken links, unused/unreferenced files, and untagged documents.



\subsection{Results}
 
 
 
\section{Byproducts}

\subsection{ADCs}

For such an analog signal processing project, a Digital Acquisition (DAQ) tool was required. Most AFEs do include ADCs, but starting from scratch required efficient and fast ADCs. 

However, due to the speed of acquisition higher than raw data throughput exceed a around 230MB/s, real time was never really achieved and corresponding storage buffer was needed. Several options where explored, including using a BeagleBone Black DAQ at 40Msps, a STM32 onboard DAQ up to 7.2Msps. Finally, the choice was to develop a high-speed acquisition extension for a omnipresent raspberry pi.

The limitation of the raspberry pi meant that the duration of captures was limited, and that acquisition was limited to around 22Msps, being the rate to which GPIO state was copied to memory. 

Interestingly enough, such a polyvalent byproduct was deemed of interest by some researchers.



\subsection{phantoms}

\cite{nolting_inexpensive_2016} from SPAM

\subsection{Suppliers}


\subsection{Materials}


On most designs, an acoustic window is needed to seal the scanhead while minimizing signal loss.

An oftne used material is the TPX (polymethylpentene), used for example on an hand-held high frequency ultrasound scanner  \cite{erickson_hand-held_2001} or on another bimorph design \cite{brown_low_2013}. In the latter, the scanhead is located inside a 3D printed plastic probe housing and filled with de-ionized water.

\section{Open source hardware community}

\subsection{Hackaday}


\subsection{ice40}



\subsection{Business-like}



\subsubsection{Certification}



\subsubsection{Tindie}




\subsection{Slack}

Communication is key for open-source projects. Part of it goes through documentation, which enables the constitution of an asynchronous base of knowledge. 

However, real-time chat tools, such as slack, is useful for real time interactions, calls around a community, ...


\section{Ease of Use}

\subsection{Maintaining the Integrity of the Specifications}

The IEEEtran class file is used to format your paper and style the text. All margins, 
column widths, line spaces, and text fonts are prescribed; please do not 
alter them. You may note peculiarities. For example, the head margin
measures proportionately more than is customary. This measurement 
and others are deliberate, using specifications that anticipate your paper 
as one part of the entire proceedings, and not as an independent document. 
Please do not revise any of the current designations.

\section{Prepare Your Paper Before Styling}
Before you begin to format your paper, first write and save the content as a 
separate text file. Complete all content and organizational editing before 
formatting. Please note sections \ref{AA}--\ref{SCM} below for more information on 
proofreading, spelling and grammar.

Keep your text and graphic files separate until after the text has been 
formatted and styled. Do not number text heads---{\LaTeX} will do that 
for you.

\subsection{Abbreviations and Acronyms}\label{AA}
Define abbreviations and acronyms the first time they are used in the text, 
even after they have been defined in the abstract. Abbreviations such as 
IEEE, SI, MKS, CGS, ac, dc, and rms do not have to be defined. Do not use 
abbreviations in the title or heads unless they are unavoidable.

\subsection{Units}
\begin{itemize}
\item Use either SI (MKS) or CGS as primary units. (SI units are encouraged.) English units may be used as secondary units (in parentheses). An exception would be the use of English units as identifiers in trade, such as ``3.5-inch disk drive''.
\item Avoid combining SI and CGS units, such as current in amperes and magnetic field in oersteds. This often leads to confusion because equations do not balance dimensionally. If you must use mixed units, clearly state the units for each quantity that you use in an equation.
\item Do not mix complete spellings and abbreviations of units: ``Wb/m\textsuperscript{2}'' or ``webers per square meter'', not ``webers/m\textsuperscript{2}''. Spell out units when they appear in text: ``. . . a few henries'', not ``. . . a few H''.
\item Use a zero before decimal points: ``0.25'', not ``.25''. Use ``cm\textsuperscript{3}'', not ``cc''.)
\end{itemize}

\subsection{Equations}
Number equations consecutively. To make your 
equations more compact, you may use the solidus (~/~), the exp function, or 
appropriate exponents. Italicize Roman symbols for quantities and variables, 
but not Greek symbols. Use a long dash rather than a hyphen for a minus 
sign. Punctuate equations with commas or periods when they are part of a 
sentence, as in:
\begin{equation}
a+b=\gamma\label{eq}
\end{equation}

Be sure that the 
symbols in your equation have been defined before or immediately following 
the equation. Use ``\eqref{eq}'', not ``Eq.~\eqref{eq}'' or ``equation \eqref{eq}'', except at 
the beginning of a sentence: ``Equation \eqref{eq} is . . .''

\subsection{\LaTeX-Specific Advice}

Please use ``soft'' (e.g., \verb|\eqref{Eq}|) cross references instead
of ``hard'' references (e.g., \verb|(1)|). That will make it possible
to combine sections, add equations, or change the order of figures or
citations without having to go through the file line by line.

Please don't use the \verb|{eqnarray}| equation environment. Use
\verb|{align}| or \verb|{IEEEeqnarray}| instead. The \verb|{eqnarray}|
environment leaves unsightly spaces around relation symbols.

Please note that the \verb|{subequations}| environment in {\LaTeX}
will increment the main equation counter even when there are no
equation numbers displayed. If you forget that, you might write an
article in which the equation numbers skip from (17) to (20), causing
the copy editors to wonder if you've discovered a new method of
counting.

{\BibTeX} does not work by magic. It doesn't get the bibliographic
data from thin air but from .bib files. If you use {\BibTeX} to produce a
bibliography you must send the .bib files. 

{\LaTeX} can't read your mind. If you assign the same label to a
subsubsection and a table, you might find that Table I has been cross
referenced as Table IV-B3. 

{\LaTeX} does not have precognitive abilities. If you put a
\verb|\label| command before the command that updates the counter it's
supposed to be using, the label will pick up the last counter to be
cross referenced instead. In particular, a \verb|\label| command
should not go before the caption of a figure or a table.

Do not use \verb|\nonumber| inside the \verb|{array}| environment. It
will not stop equation numbers inside \verb|{array}| (there won't be
any anyway) and it might stop a wanted equation number in the
surrounding equation.

\subsection{Some Common Mistakes}\label{SCM}
\begin{itemize}
\item The word ``data'' is plural, not singular.
\item The subscript for the permeability of vacuum $\mu_{0}$, and other common scientific constants, is zero with subscript formatting, not a lowercase letter ``o''.
\item In American English, commas, semicolons, periods, question and exclamation marks are located within quotation marks only when a complete thought or name is cited, such as a title or full quotation. When quotation marks are used, instead of a bold or italic typeface, to highlight a word or phrase, punctuation should appear outside of the quotation marks. A parenthetical phrase or statement at the end of a sentence is punctuated outside of the closing parenthesis (like this). (A parenthetical sentence is punctuated within the parentheses.)
\item A graph within a graph is an ``inset'', not an ``insert''. The word alternatively is preferred to the word ``alternately'' (unless you really mean something that alternates).
\item Do not use the word ``essentially'' to mean ``approximately'' or ``effectively''.
\item In your paper title, if the words ``that uses'' can accurately replace the word ``using'', capitalize the ``u''; if not, keep using lower-cased.
\item Be aware of the different meanings of the homophones ``affect'' and ``effect'', ``complement'' and ``compliment'', ``discreet'' and ``discrete'', ``principal'' and ``principle''.
\item Do not confuse ``imply'' and ``infer''.
\item The prefix ``non'' is not a word; it should be joined to the word it modifies, usually without a hyphen.
\item There is no period after the ``et'' in the Latin abbreviation ``et al.''.
\item The abbreviation ``i.e.'' means ``that is'', and the abbreviation ``e.g.'' means ``for example''.
\end{itemize}

An excellent style manual for science writers is   .

\subsection{Authors and Affiliations}
\textbf{The class file is designed for, but not limited to, six authors.} A 
minimum of one author is required for all conference articles. Author names 
should be listed starting from left to right and then moving down to the 
next line. This is the author sequence that will be used in future citations 
and by indexing services. Names should not be listed in columns nor group by 
affiliation. Please keep your affiliations as succinct as possible (for 
example, do not differentiate among departments of the same organization).

\subsection{Identify the Headings}
Headings, or heads, are organizational devices that guide the reader through 
your paper. There are two types: component heads and text heads.

Component heads identify the different components of your paper and are not 
topically subordinate to each other. Examples include Acknowledgments and 
References and, for these, the correct style to use is ``Heading 5''. Use 
``figure caption'' for your Figure captions, and ``table head'' for your 
table title. Run-in heads, such as ``Abstract'', will require you to apply a 
style (in this case, italic) in addition to the style provided by the drop 
down menu to differentiate the head from the text.

Text heads organize the topics on a relational, hierarchical basis. For 
example, the paper title is the primary text head because all subsequent 
material relates and elaborates on this one topic. If there are two or more 
sub-topics, the next level head (uppercase Roman numerals) should be used 
and, conversely, if there are not at least two sub-topics, then no subheads 
should be introduced.

\subsection{Figures and Tables}
\paragraph{Positioning Figures and Tables} Place figures and tables at the top and 
bottom of columns. Avoid placing them in the middle of columns. Large 
figures and tables may span across both columns. Figure captions should be 
below the figures; table heads should appear above the tables. Insert 
figures and tables after they are cited in the text. Use the abbreviation 
``Fig.~\ref{fig}'', even at the beginning of a sentence.

\begin{table}[htbp]
\caption{Table Type Styles}
\begin{center}
\begin{tabular}{|c|c|c|c|}
\hline
\textbf{Table}&\multicolumn{3}{|c|}{\textbf{Table Column Head}} \\
\cline{2-4} 
\textbf{Head} & \textbf{\textit{Table column subhead}}& \textbf{\textit{Subhead}}& \textbf{\textit{Subhead}} \\
\hline
copy& More table copy$^{\mathrm{a}}$& &  \\
\hline
\multicolumn{4}{l}{$^{\mathrm{a}}$Sample of a Table footnote.}
\end{tabular}
\label{tab1}
\end{center}
\end{table}

\begin{figure}[htbp]
\caption{Example of a figure caption.}
\label{fig}
\end{figure}

Figure Labels: Use 8 point Times New Roman for Figure labels. Use words 
rather than symbols or abbreviations when writing Figure axis labels to 
avoid confusing the reader. As an example, write the quantity 
``Magnetization'', or ``Magnetization, M'', not just ``M''. If including 
units in the label, present them within parentheses. Do not label axes only 
with units. In the example, write ``Magnetization (A/m)'' or ``Magnetization 
\{A[m(1)]\}'', not just ``A/m''. Do not label axes with a ratio of 
quantities and units. For example, write ``Temperature (K)'', not 
``Temperature/K''.

\section*{Acknowledgment}

The preferred spelling of the word ``acknowledgment'' in America is without 
an ``e'' after the ``g''. Avoid the stilted expression ``one of us (R. B. 
G.) thanks $\ldots$''. Instead, try ``R. B. G. thanks$\ldots$''. Put sponsor 
acknowledgments in the unnumbered footnote on the first page.

\section*{References}

Please number citations consecutively within brackets . The 
sentence punctuation follows the bracket  . Refer simply to the reference 
number, as in ---do not use ``Ref.  '' or ``reference  '' except at 
the beginning of a sentence: ``Reference was the first $\ldots$''

Number footnotes separately in superscripts. Place the actual footnote at 
the bottom of the column in which it was cited. Do not put footnotes in the 
abstract or reference list. Use letters for table footnotes.

Unless there are six authors or more give all authors' names; do not use 
``et al.''. Papers that have not been published, even if they have been 
submitted for publication, should be cited as ``unpublished'' . Papers 
that have been accepted for publication should be cited as ``in press''. 
Capitalize only the first word in a paper title, except for proper nouns and 
element symbols.

For papers published in translation journals, please give the English 
citation first, followed by the original foreign-language citation \cite{megalingam_fpga_2016}.

\bibliographystyle{ieeetr}
\bibliography{all} 



\end{document}
