\documentclass[conference]{IEEEtran}
\IEEEoverridecommandlockouts
% The preceding line is only needed to identify funding in the first footnote. If that is unneeded, please comment it out.
\usepackage{cite}
\usepackage{amsmath,amssymb,amsfonts}
\usepackage{algorithmic}
\usepackage{graphicx}
\usepackage{textcomp}
\usepackage{xcolor}
\def\BibTeX{{\rm B\kern-.05em{\sc i\kern-.025em b}\kern-.08em
    T\kern-.1667em\lower.7ex\hbox{E}\kern-.125emX}}
\begin{document}

\title{The interest of single-sensor ultrasound imaging}

\author{\IEEEauthorblockN{Luc Jonveaux}
\IEEEauthorblockA{\textit{Open source tinkerer} \\
\textit{Independant}\\
Milly, France\\
contact@un0rick.cc}

}

\maketitle

\begin{abstract}
This document is a model and instructions for \LaTeX.
This and the IEEEtran.cls file define the components of your paper [title, text, heads, etc.]. *CRITICAL: Do Not Use Symbols, Special Characters, Footnotes, 
or Math in Paper Title or Abstract.
\end{abstract}

\begin{IEEEkeywords}
ultrasound imaging, open source, hardware, ultrasound
\end{IEEEkeywords}


\section{Principles of ultrasound}


\subsection{Physics}


\subsection{Modes}





\subsubsection{A-Mode imaging}


This modality can be used for example in image free systems for evaluation of vascular stiffness. \cite{joseph_artsenstouch_2015}


Probes \cite{carotenuto_very_2004}
Systems

but these have not been open-sourced.



Even if relatively simple, the A-mode enables not imaging, but rather measurements, enabling the examination of paranasal sinuses, transkull fluid detection, sinus pathology, skeletal muscle detection in the guided wrist extension, automatic measurement of lumen diameter of carotid artery \cite{li_new_2014}, or bone porosit \cite{wahab_design_2016}.


\subsubsection{B-mode}

\subsubsection{C-mode}

\subsubsection{Doppler}

\subsubsection{Tomography}

\cite{kuzmin_fast_2016}

\subsection{Architecture}

\subsubsection{specs from who}


% iris.wpro.who.int/bitstream/handle/10665.1/5467/9290611219_eng.pdf

\cite{kurjak_use_1986}


In 1986, the specifications for a General Purpose Ultrasound Scanner were as follows:

- Have linear and sector types
- Frequency of interest is 3.5MHz, possibly extensible to 5MHz.
- Sector angle of 40 degrees or more
- Depth of 18cm
- rate of 5 to 10Hz for sector scanning
- image to be frozen, displayed on a 512x512 image on 4 bits.

Minimal specifications for hardware can be infered:

- An ADC at at least 10MHz, running on 8bits
- 18cm means 240us per acquisition, or 2400 points at 10Msps. Details can be around 0.5mm, that's then  480 points per line. Let's consider 512 for the image size.
- Let's consider 60 degree. To have a 1mm resolution at 18cm means 180 lines per image, which will be fed into the 512x512 image. However, given the size of the image and characterics of the piezo, it seems reasonable to have 60 lines over 40 degrees.
- Raw images are then 60x512x8bits, that's a memory of 30kB+ before scan conversion.
- 60 lines of 240us means 14ms ofr an image, capping the framerate to close to 70fps - seems fine. However, mechanical overheads reduce this figure.

\subsubsection{Arrays}


However, the need heavier setup, for example to use up to 128 channels setups \cite{assef_flexible_2015}.

\subsubsection{Others}

Configurable hardware makes the system resilient to future changes.
Beyond this hardware designs can be adjusted without reprinting the circuit board.

FPGAs improve the ability for ultrasound imaging systems to create small form factor and high-performance products with reduced power consumption \cite{dusa_low_2014}. In the latter case, it enables the control of a 8 channels platform.
  
Others use  an  additional microcontroler is set up between the FPGA and the USB and provide configuration on the fly, ,providing access to the platform through Matlab. It can interface to an array using several Octal Ultrasound ADC/LNA/VGA/AAF ( AD9272 ) \cite{raj_microcontroller_2017}, \cite{raj_8051_2016}.

\section{Uses}

\subsection{Medical imaging}

\subsubsection{Ultrasound as part of medical imaging }

Ultrasound is:

* cost-efficiency. Signal generation uses a piezoelectric material, requiring much less than setting up for example a large magnetic field for an MRI scan. 
* relatively safe: as there are no ionizing radiation, ultrasound devices do not need special safety equipment like shields, or expensive supplies.
* light: on the sense that compared to other imaging modalities, devices can be made relatively portable
* dynamic


See Kurjak


\subsection{NDT}



\cite{fritsch_full_nodate} presents a very interesting design for single element FPGA-based NDE design. It migrates traditionally analog functions, like filtering and envelope extraction to   the   digital   domain, as well as includes an EMI filteR. An interesting data compression scheme is also presented.

\subsection{Other}

\subsubsection{Ultrasound capsule}

Swallowable capsule : capsule endoscopy using  high frequency microultrasound (µUS). This is a great example of miniaturisation fitting all in a  10 mm diameter ×30 mm long capsule , with all the constraints discussed above \cite{cox_ultrasound_2017}.

further develops mechanics used in such a capsule \cite{wang_development_2017}.

\subsubsection{Fingers movement}

\cite{sikdar_novel_2014}

\subsubsection{Sleep apnea}


\cite{weng_fpga-based_2015}

\subsubsection{Biofeedack}


 
\section{Litterature review}

\subsection{Architectures}

%% some benchmark now


\begin{center}
 \begin{tabular}{||c | c c c c c | c ||} 
 \hline
 Paper 			& Elements 	& Voltage 	& ADC 			& AFE / TGC 	& Memory 	& Others\\ [0.5ex] 
 \hline\hline	
 assef_design_2012	& 128		& 100 Vpp 	& 50Msps @ 12 bits	& AFE5805	& ? 		& BF: MD2131 \\ 
 \hline
 assef_compact_2014	& 128		& 100 Vpp  	& 50Msps @ 12 bits	& AFE5805	& 256 MB SDRAM  & BF: MD2131 \\ 
 \hline
 assef_flexible_2015	& ? 		& 100 Vpp 	& 40Msps @ 12 bits 	& AFE5805	&  ?		& BF: MD2131 \\ 
 \hline
 bharath_portable_2015	& ?	 	& ?	 	& ?			& ?		& 256 MB RAM	& RPi compatible\\ 
 \hline
 bharath_novel_2016	& 8	 	& +-50V	 	& 40Msps @ 12 bits	& AFE5808	& ? 		& MAX14808 \\ 
 \hline
 chatar_analysis_2016	& 16	 	& ?	 	& 150MSPS 14-bit	& ?		& 256 MB RAM	& n/a \\ 
 \hline
 dusa_low_2014		& 8 		& 100 Vpp 	& 65MSps @ 12 bits 	& AFE5809	& ? 		& n/a \\ 
 \hline
 govindan_reconfigurable_2015	& 8	 & ?	 	& 250 MSps @ 8 bits	& VCA8500	& ? 		& BF: LM96551, PLS: LM96551\\ 
 \hline
 jonveaux_arduino-like_2017	& Single & 100Vpp 	& 22Msps @ 9 bits	& TGC: AD8331	& ? 		& Using ADL5511 \\ 
 \hline
kushi_ultrasonic_2017	& Single	& ?		& 100Msps @ 14 bits 	& ?		& 256 MB RAM	& n/a \\
 \hline
 kim_smart-phone_2017	& 128 (32 ch)	& +-80 V 	& 40MSps @ 12 bits 	& ?		& ? 		& Smartphone use\\ 
 \hline
 kruizinga_compressive_2017 & Single 	& 100 Vpp 	& 200MSps @ 12 bits 	& ?		& ?		& n/a \\
 \hline
 lee_new_2014		& 16 		& ?		& 40Msps @		& AFE5808	& ? 		& MAX14808\\ 
 \hline
 li_new_2014		& 1	 	& 80 V	 	& 40MSps @ 12 bits 	& AD9276	& ?		& n/a \\
 \hline
 peyton_comparison_2018	& 32	 	& ?	 	& 2x10Msps, ADC10D020	& Custom	& ? 		& ADM7155\\ 
 \hline
 qiu_delayed-excitation_2018 	& Single  & +48V (TC6320) & 160Msps		& TGC: AD8331	& 2Gb		& RAM: MT41J128M16HA \\ 
 \hline
 cheung_multi-channel_2012 & 	128 	& ?		& 40Msps, 12 bits / 80Msps @ 10 bits & AD9272 & 16 GB 	& TX810 as TX/RX	
 \hline
 wall_high-_2010	& ? 		& 12 V		& 65 Msps		& ? 		& ?		& ? 
 \hline
 ibrahim_towards_2018	& 64		& 12 V		& 20 Msps, 12 bits	& ? 		& ?		& Kintex UltraScale KU040
 \hline
 ricci_programmable_2006 & 1		& 100 V		& 64 Msps, 14 bits	& MAX4107 + AD603 & 64Mbytes	& ?
 \hline
 hager_ultralight:_2017 & 64		& 100Vpp (HV7350) 	& 32.5 Msps @ 12 bit & AFE5851	& ?		& HV: ADP1614/ADP1621
 \hline
 @todo Qiu2011 		& 1		& ? 		& 80Msps @ 10 bit  & SMA231 +THS4509 	& 512 Mb DDRAM	& ADC up to  200Msps (ADS5517)
 \hline
 fritsch_full_nodate	& 1		& 50-400V 	& 80Msps   		& ?  		& ?	& XC2S200
 \hline
 weng_fpga-based_2015	& 16		& 100V 	& 150 Msps @ 10 bits   		& Max2077 	& ?	& Pulser: HV7350, AD9254, SMBD7000



\end{tabular}
\end{center}





\subsection{Mechanics}


\subsubsection{Sweeping means}


mechanical sweeping example \cite{svilainis_electronics_2014}.


One would note however, that for power-uses, plane wave imaging, made possible because of GPUs \cite{hewener_mobile_2015}, would limit the formation of an image to a single pulse/receive event.


With high frequency transducers (c. 50 MHz), imaging transducers are relatively smaller, which makes mechanical scanning a great solution. This implies however strong positionning control and motors \cite{carotenuto_very_2004}. In this article, a piezoelectric	 motor is used in conjunction with an optical encoder. 

\subsection{Materials}


On most designs, an acoustic window is needed to seal the scanhead while minimizing signal loss.

An oftne used material is the TPX (polymethylpentene), used for example on an hand-held high frequency ultrasound scanner  \cite{erickson_hand-held_2001} or on another bimorph design \cite{brown_low_2013}. In the latter, the scanhead is located inside a 3D printed plastic probe housing and filled with de-ionized water.

\subsection{Smartphone capabilities}

\cite{kim_smart-phone_2017} proposes a design usable on a smartphone, connected through USB. Plane-wave acquisition, mentioned earlier \cite{hewener_mobile_2015} also enables smarter captures.


 
Proof of concept for a portalbe 32-channel ultrasound unit \cite{kim_single_2012}

\subsection{Improvements}
 
\subsubsection{MSAS/MFFS}



\cite{ylitalo_ultrasound_1994}

\cite{heuvel_development_2017}

\cite{nikolov_fast_2008}


In 1974, \cite{burckhardt_experimental_1974} showed that using a synthetic aperture design as an hybrid  between holography and B scan offers a higher lateral resolution as well as better detection of tilted specular reflectors.

\subsubsection{Time reversal focusing}

@adapt N.Etaix et.al [19] presented an idea of a low-profile acoustic imaging device with only one transmit/receive element. According to the theory, if the medium is reciprocal and the channel’s impulse response is known, the latter can be re-emitted in time-reversed order. Mathematically, this results in the response auto-convolution [20]. Knowing that the auto-convolution has a peak in the origin, focusing is effectively achieved. The time reversal result is equivalent to matched filtering – energy maximization at the desired location in space and time.

\subsubsection{Compressed sensing}

Traditional 2-D and 3D ultrasound require the use of hundreds of delicates sensors, with matching hardware. 

Ultrasound 2D arrays, or matrices, have developped available, however this also require more hardware (AFE, ADC units, cabling, etc..) and hence become more
complex and expensive to produce, and require heavy data processing.

@todo add \cite{Fedjajevs_mscthesis_14_09_2016} thesis

It appears that classical sampling is challenged by the signal processing "compressive sensing" field. This shows that most signals have a sparse representation - a finite sparse signal can be reconstructed from a small set of linear, non-adaptive measurements.

The recent development of compressive sensing shows that one can encode individual voxels through a 'chaotic' intermediary (eg a lense), and allows to design simple ultrasound imaging equipment that can provide 3D imaging using a single-element ultrasound sensor.  

Imaging is then offloaded to the processing domain supported by constantly growing electronic devices' capabilities (memory, GPU's).


This open doors to simpler hardware - and new applications \cite{kruizinga_compressive_2017}.


\cite{hua_compressed_2011}

\subsubsection{3D enhancement}
 
see augmented \cite{herickhoff_low-cost_2018}

or 

\cite{poulsen_optical_2005.pdf}

or 

\cite{lorenzo_experimental_2009} pour reperer dans l'espace


\subsubsection{Enveloppe detection} 


For FPGA
\cite{chang_novel_2007}



\subsubsection{Data reduction}
 
 
\cite{soto-cajiga_fpga-based_2012}

\section{Open-source designs}

\subsection{Past works}

\subsection{Modules}


Previous work was done on an arduino-like approach \cite{jonveaux_arduino-like_2017}.

\subsection{FPGA board}

\subsubsection{Why FPGAs ?}

\cite{zhang_fpga_2012}


\subsection{Forks}

\section{Lessons in open-source}

\subsection{Medical imaging and open source}

\subsubsection{Open source medical hardware}

Open hardware lowers some barriers: having a full design under an open-source license provides the key to all to contribute and improve a design. This in turns allows a possible rapid spread of the design, customisation for specific uses, and ad-hoc modification. A design being open means that a higher number of contributors can help inspect and improve it.

Repairs are also easier with the code, with several types of impact (social, environmental and economics) that enhance the lifecycle of a product.

The open source movement has been facilitated by the numerous platforms and communication means, that allow these designs to be copied, but also for robust storage and distribution means. 


\cite{niezen_open-source_2016}


Open source hardware can be a disruptive tool on the medical device market. Of course, a caveat has to be made on the non-certification of most of these devices, which in turn would hint at open-source being a major tool in innovation, leaving the industrialisation of such devices to professionals. Shorter development cycles, even for hardware, with open source permits quick iterations over a product.


A study by Niezen \cite{niezen_open-source_2016} lists a selection of open-source medical projects. Most are working with Arduino or Raspberry Pi, showing the importance of existing projects and the way open-source allows for reuse of previous projects. Dedicated hardware is developped for this project, leaving the specifics of the project to be developped by the contributor.

\subsection{Documenting}

\subsubsection{How to document}

On this project, a set of documenting tools was created. Documentation is based on the constitution of a project body of knowledge, which consists in code files, documentation files, images, and other files being tagged with structured information.

The concept used has been to store information only once, and closest to the source of the documentation. The main objective is to avoid redundancy and to allow for tracking outdated information, for elimination.

Tags cover for example date and time, type of file, experiment related to the file, probes being used, description of the file, author, ...  "Template files", that have "includes" of parts of documentation, also dynamically generate files.

This way, a script parses the whole, possibly unstructured repository and builds a body of knowledge, which in turn is used to generate static pages, that can cover:

- List of experiment and descriptions
- List of probes
- List of modules
- Jupyter notebooks
- Readme pages

The structured data and produced static pages are also pushed on a gitbook, so that documentation is easily accessible and searchable. Even Jupyter notebooks are converted to markdown for ease of browsing.

Moreover, a worklog is used to generate a lab-log of research, published on a github-page structure. Presentations are dynamically generated as well from markdown files, and a log of the documentation compilation is generated, tracking broken links, unused/unreferenced files, and untagged documents.


\subsection{Communities}

\subsubsection{Hackaday}

\subsubsection{ice40}

\subsubsection{Tindie}

\subsubsection{Slack}

Communication is key for open-source projects. Part of it goes through documentation, which enables the constitution of an asynchronous base of knowledge. 

However, real-time chat tools, such as slack, is useful for real time interactions, calls around a community, ...

\subsection{Suppliers}

\subsection{An example of byproducts: ADCs}

For such an analog signal processing project, a Digital Acquisition (DAQ) tool was required. Most AFEs do include ADCs, but starting from scratch required efficient and fast ADCs. 

However, due to the speed of acquisition higher than raw data throughput exceed a around 230MB/s, real time was never really achieved and corresponding storage buffer was needed. Several options where explored, including using a BeagleBone Black DAQ at 40Msps, a STM32 onboard DAQ up to 7.2Msps. Finally, the choice was to develop a high-speed acquisition extension for a omnipresent raspberry pi.

The limitation of the raspberry pi meant that the duration of captures was limited, and that acquisition was limited to around 22Msps, being the rate to which GPIO state was copied to memory. 

Interestingly enough, such a polyvalent byproduct was deemed of interest by some researchers.


\subsection{Phantoms and gel}


\subsubsection{Phantoms}

\cite{nolting_inexpensive_2016} from SPAM

\subsubsection{Gel}

\subsection{Open source certification}

\section{Next steps}



\section*{Acknowledgment}

Thanks guys

%\section*{References}



\bibliographystyle{ieeetr}
\bibliography{all} 



\end{document}
